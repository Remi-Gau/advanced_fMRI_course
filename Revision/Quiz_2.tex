% $Header: /home/vedranm/bitbucket/beamer/solutions/generic-talks/generic-ornate-15min-45min.en.tex,v 90e850259b8b 2007/01/28 20:48:30 tantau $

\documentclass{beamer}

\mode<presentation>
{
\usetheme{Warsaw}
\setbeamercovered{transparent}
}

\usepackage[english]{babel}
\usepackage[latin1]{inputenc}
\usepackage{times}
\usepackage[T1]{fontenc}


\title[MRes Advanced Brain Imaging]
{MRes Advanced Brain Imaging}

\subtitle[Revision session]
{Revision session}

\author[Remi Gau]
{Remi Gau}

\institute[University of Birmingham]
{
School of psychology\\
University of Birmingham
}

\date[Short Occasion]
{17\textsuperscript{th} March 2014}

\subject{MRes Advanced Brain Imaging - Revision session}

\pgfdeclareimage[height=0.5cm]{university-logo}{university-logo-filename.jpeg}
\logo{\pgfuseimage{university-logo}}

% If you wish to uncover everything in a step-wise fashion, uncomment
% the following command: 
% \beamerdefaultoverlayspecification{<+->}


\begin{document}

\footnotesize


\begin{frame}
  \titlepage
\end{frame}

%----------------------------------------------------------------------------------------------------------------------------------------------------------
%----------------------------------------------------------------------------------------------------------------------------------------------------------

\section{PRE-PROCESSING}

%----------------------------------------------------------------------------------------------------------------------------------------------------------

\subsection[General questions]{General questions}


\begin{frame}{General questions}
  \begin{itemize}
    \item \textbf{How do you map from voxel space to world space?}

    \smallskip 
%     You multiply the transformation matrix with the vector containing the voxel coordinates.

    \item \textbf{What are the real world coordinates of the voxel with indices $\left[\begin{array}{ccc} 1&1&1\end{array}\right]$ of an image with the following tranformation matrix?}

    \begin{center}
      $
      \left[
      \begin{array}{cccc}
      1 & 0 & 0 & 25\\
      0 & 1 & 0 & 57\\
      0 & 0 & 1 & 45\\
      0 & 0 & 0 & 1\\
      \end{array}\documentclass[a4paper,10pt]{article}

\usepackage{ucs}
\usepackage[utf8]{inputenc}
\usepackage{babel}
\usepackage{fontenc}
\usepackage{graphicx}

\usepackage[dvips]{hyperref}

\date{28/09/2014}

\begin{document}
                  
                 \end{document}

      \right]
      $
    \end{center}

  \end{itemize}
\end{frame}


\begin{frame}{General questions}
  \begin{itemize}
    \item \textbf{What is an objective function when comparing 2 images?}

    \smallskip 
%     It is a function that allows quantification of the difference between these images and the optimization of the parameters set to get their best realignment, usually by minimizing/maximizing its value.

    \bigskip
    \item \textbf{Why are there different objective functions?}

    \smallskip   
%     T1 and T2* have very different ranges of values for the same tissue class. So intermodal and intramodal coregistration are going to require different types of way of quantifying misalignment.
  \end{itemize}
\end{frame}
\documentclass[10pt,draft]{beamer}

\usepackage{ucs}
\usepackage[utf8]{inputenc}
\usepackage{beamerthemebars}

\usepackage{babel}
\usepackage{fontenc}
\usepackage{graphicx}

\date{28/09/2014}

\begin{document}
 
\end{document}


\end{document}


